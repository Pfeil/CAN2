
\section{Produktfunktionen}

\subsection{Funktionale Anforderungen}
\begin{enumerate}[{/FA}1{/}]
\item Neuen Song erstellen
\item Song Attribut ändern (Titel, Interpret, $\dots$)
\item Song löschen
\item Neuer Anhang
\item Anhang löschen
\item Song Tags editieren
\item Anhänge:
\begin{itemize}
	\item Audioanhang
	\item PDF Anhang
	\item Textanhang
\end{itemize}
\item Audio- und PDF Anhang werden automatisch erstellt, wenn passende mp3/ogg/pdf/... gefunden wird.
\item Audioanhang kann Audiodateien wiedergeben
\item PDF Anhang kann PDF anzeigen, zoomen, Seite wählen
\item Textanhang speichert Text
\item Textanhang erkennt Akkorde
\item Textanhang empfängt Text per Copy\&Paste. Dabei werden Tabulatoren automatisch in (ggf. mehrere) Leerzeichen umgewandelt.
\item Textanhang kann Transponieren
\item Neuen Termin erstellen
\item Termin mit Songs verknüpfen
\item Termin editieren
\item PDF exportieren (mit vielen Optionen: Welche Anhänge, In-/Exklusive Tags, Formatierung, $dots$ )
\end{enumerate}

\begin{enumerate}[{/WA}1{/}]
\item Audiodateien mit variabler Geschwindigkeit abspielen
\item Bestimmten Ausschnitt der Audiodatei endlos wiederholen
\item Anzeige Modus anbieten
\item Android/iOS-Portierung
\end{enumerate}


\subsection{Groupware}

An einem Projekt sollen beliebig viele Benutzer (gleichzeitig) arbeiten können. Es gibt verschiedene Ansätze. 

\subsubsection{Datenbank}
\begin{itemize}
\item Vorteile:
	\begin{itemize}
	\item weite Verbreitung
	\item einfach zu hosten
	\item viele Bibliotheken und Kompatibilitäten
	\end{itemize}
\item Nachteile:
	\begin{itemize}
	\item Komplexes caching und conflict management oder Onlinezwang
	\item Unflexibles Datenformat
	\end{itemize}
\end{itemize}

\subsection{Git}
\begin{itemize}
\item Vorteile:
	\begin{itemize}
	\item frei verfügbar
	\item einfaches und kostenloses Hosting bei z.B. GitHub
	\item Arbeiten offline möglich, nur push/pull benötigen Onlinezugang
	\item conflict managment
	\end{itemize}
\item Nachteile:
	\begin{itemize}
	\item Windows-Abhängigkeit: MinGW (großer Installer, benötigt viel Platz)
	\item Kann nicht mit Binärdateien umgehen
	\item Benutze libgit2. Bibliothek verträgt sich nicht mit Qt
	\item Manuelle Konfliktbehebung evtl. nötig
	\end{itemize}
\end{itemize}

Deshalb: Lagere Datei und Versionsverwaltung in externe Bibliothek aus.

Dazu wird die Bilbliothek GitZip benuzt.
Ein Projekt besteht dabei für den Benutzer transparent als \emph{einzelne} Datei gespeichert.
Beim lesen wird die Datei in ein temporäres Arbeitsverzeichnis entpackt (z.B. zip).

Git arbeitet ebenfalls auf dem temporären Arbeitsverzeichnis.
Beim speichern wird das Verzeichnis gepackt und in die zip Datei in das vom Benutzer festgelegte Verzeichnis geschrieben.

















