\section{Zielbestimmung}

CAN ist ein Programm, dass Coverbands bei der Organisation der Auftritte, Proben, Repertoire, etc. Unterstützen soll.

\subsection{Musskriterien}

\begin{enumerate}[{/MK}1{/}]
\item Neuen Song hinzufügen
\item Song löschen
\item Song editieren
\item Song taggen
\item Songs als PDF Export (Akkordschema, Songtext, Anhang, $\dots$), siehe \ref{PDFExport}
\item Termine verwalten (Probe, Auftritt)
\item Probeplan verwalten
\item Integrierte, einfache Versionskontrolle, Backup und Synchronization (\emph{Groupware}, siehe \ref{groupware})
\item Youtube Downloader, siehe \ref{youtube}
\end{enumerate}

\subsection{Wunschkriterien}
\begin{enumerate}[{/WK}1{/}]
\item\label{wkmusiccollection} Verknüpfung mit Musiksammlung 
	\begin{itemize}
	\item Versuche bei Programmstart Song aus Repertoire in Musiksammlung zu finden. Bei Erfolg, Verknüpfung herstellen.
	\end{itemize}
\item Verknüpfte Musikdateien können abgespielt werden.
\item Verknüpfung mit Notensammlung, siehe Punkt \ref{wkmusiccollection}
\item Verknüpfte PDF-Dateien können angezeigt werden
\end{enumerate}

\subsection{Abgrenzungskriterien}
\begin{enumerate}[{/AK}1{/}]
\item Kein Audioeditor
\item Kein Formatwandler
\item Keine Videos
\end{enumerate}

